\documentclass[12p]{article}
\usepackage{amsmath}
\usepackage{mathtools}

\newcommand{\logicarg}[2]{
  \begin{tabular}[t]{@{}l@{}}
    #1 \\ \hline #2 
  \end{tabular}
}
\begin{document}
\section{manejo de errores}

errores y la maquina k 
Hay dos maneras de manejar errores. Estas son con los enunciados catch y con los enunciados handle.
Los enunciados catch y handle se son utilizados cuando el enunciado se evalua utilizando la maquina K.
Ahora, recuerda que las transiciones de la maquina k se parecian a la forma en la que se expresaba la semantica dinamica.

pequeno recordatorio de los estados y semantica dinamica de la maquina k 
la maquina k va a poder estar en 3 estados(En un principio son 2 , el de evaluar y devolver .Sin embargo ,cuando se quiere extender para
que sea posible manejar errores , es necesario anadir un tercero).Donde cada estado esta especificado por una pila y una expresion. 
el significado de cada uno de los estados es el siguiente: \\ 
\begin{enumerate}

\item $ P \gg e $ significa que la expresion e se va a evaluar con la pila P
\item $ P \ll e $ significa que la expresion e se esta devolviendo. Con la pila P

\item $ P \ll e $ es un estado especial . Este estado es para propagar el error y no permitir que se siga evaluando algo.
\end{enumerate}

sin manejo y con manejo ...
Ahora, puedes tratar con los errores de dos maneras : La primera es sin manejar el error ,es decir la 
expresion unicamente trata a los errores como valores. Para esto debes aumentar la gramatica de la siguiente 
manera. 

como se aumenta la gramatica del lenguaje cuando no quieres que maneje los errores 

Si quieres que maneje los errores entonces hay dos formas. Con catch y con handle 
con handle puedes regresar una cadena y con catch no 
la sintaxis concreta se extiende de la siguiente manera 
entonces para que sea posible manejar los errores tienes que dejar que error sea una expresion 
ahora cuando el try trate de evaluar la expresion $ e_ 1$ y se encuentre con un error, va a tener una 
expresion para cuando esto pase que va a ser $e _2 x$ 
$e :: = \dots | error | \ try \ e _ 1 \ catch \ e_2$
la sintaxis abstracta entonces es 
$e :: = \dots | \ error \ | \ catch(e_1 , e_2 ) $
la semantica dinamica de catch tiene tres reglas 
la primera es para evaluar a catch por la izquierda hasta que sea un valor \\ 
\logicarg{$e \rightarrow e'$}{$catch(e,e_2) \rightarrow catch(e',e_2)$} \\
la segunda es para cuando la evaluacion de $e _ 1 $ fue un valor. En este caso , el bloque catch debe 
regresar el valor al que se redujo $e _ 1$ \\ 
\logicarg{}{$catch(v,e_2) \rightarrow v $} \\ 
la tercera regla es para cuando la evaluacion de $e_ 1 $ haya terminado en un error. En este caso se tiene
que proceder a evaluar a $e _2 $ para que se encargue de el error \\ 
\logicarg{}{$catch(error,e_2) -> e_2$} \\ 

Las diferencias entre el manejo de errores con catch y el manejo con handle 
La principal diferencia es que si vas a manejar un error con catch,el error no lleva nada de informacion 
osea cuando una evaluacion termina en error solo eso sabes , que hubo un error . En cambio cuando manejas 
los errores por medio de handle puedes , de alguna manera , meter informacion en el error de manera que 
cuando trates vayas a manejar el error puedas tomar informacion del error. 

\subsection{como se extiende el lenguaje para que pueda soportar a handle}
extension de la sintaxis concreta , la sintaxis abstracta , la semantica dinamica para 
poder evaluar expresiones handle

para que se pueda meter informacion en un error se debe crear un nuevo tipo de error que lleve una expresion 
dentro , de forma que cuando se lanze un error esta expresion se evalue para que pueda ser pasada como 
informacion al handle.
entonces la primera extension para la sintaxis concreta seria algo como 
$e ::= \dots | \ error e  $ y su extension en la sintaxis abstracta seria 
$e :: = \dots | \ error(e)$ 



En donde cambia con respecto a catch 
la regla de la semantica dinamica que esta relacionada con el manejo de errores en catch es de la 
siguiente forma \logicarg{}{$cathc(error,e_2) \rightarrow e2$} entonces lo que en esta regla debe cambiar 
es que ahora , dado que sabemos cual es el error , primero lo tenemos que evaluar hasta que sea un valor.
Una vez que la expresion dentro del error es evaluada tiene que ser pasada al handler . Ahora , aqui hay 
otro problema , en las expresiones $catch(e_1,e_2)$ $e_2$ podia ser cualquier expreision , pero ahora que 
el error le va a pasar informacion a la expresion que se va a encargar de el , esta expresion no puede ser 
cualquiera. Debe ser una funcion de manera que al pasar el valor este sea aplicado a la funcion que tiene  
el handler 
osea cuando tratemos de evaluar $handle(error(e_1),\lambda x. e_2)$ primero tiene que evaluar a $e_1$
\logicarg
{$e_1 \rightarrow e_1'$}
{$handle(error(e_1),\lambda x.e_2) \rightarrow handle(error(e_1',\lambda x.e_2))$}
ahora cuando dentrr del error se tenga ya un valor , este puede ser aplicado a la funcion en el handler 
que se formaliza con la siguiente regla
\logicarg{}{$handle(error(v),\lambda x.e_2 ) \rightarrow e_2[x:=v]$}

Error no tiene que ir una funcion necesariamente ya que no es la unica expresion en la que se encuentran 
variables . Recuerda que existen expresiones que pueden tener una variable ligada similares a let 
tambien como quieres tener ambos tipos para manejar errores entonces es un error ponerle a los errores el 
mismo nombre . Corrigiendo esto tenemos que para extender la sintaxis abstracta y concreata se va a hacer 
con 
la sintaxis concreta 
$e := \dots \ | raise e | handle \ e_1 \ with \ x \Rightarrow e_2$
la sintaxis abstracta 
$e : = \dots \ | raise(e) \ | handle(e_1,x.e_2) $
luego la semantica dinamica con respecto al manejo del error es la siguiente 
\logicarg{$e_1 \rightarrow e_1'$}{$handle(raise(e_1),x.e_2) \rightarrow handle(raise(e_1),x.e_2)$}
ahora cuando la expresion dentro de raise sea evaluada se va a tener 
\logicarg{}{$handle(raise(v),x.e_2) \rightarrow e_2[x:=v]$}
\end{document}