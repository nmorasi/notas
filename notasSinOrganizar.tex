\documentclass[12p]{article}
\begin{document}
git te permite mantener documentos y una historia de estos.
en dropbox sucede el siguiente problema.
supon que tienes un archivo A en el que estas trabajando y estas
trabajando en A desde una computadora $c_1$ y una computadora $c_2$.
Supon que modificas a A en la computadora $c_1$ donde tienes internet
de modo que los cambios realizados quedan en el servidor de Dropbox.
Entonces si modificas el archivo A en $c_2$ sin que antes se hayan
cargado del servidor entonces el problema es que los cambios realizados
por $c_2$ van a quedarse en el servidor y el progreso de $c_1$ va
a ser borrado.Esto puede en git evitarse con las branch. Las branch
son clones de un proyecto que se descargan localmente , permiten
que se edite algo para poder despues ,de manera segura, combinarlo
con lo que tenia el servidor.

un repositorio es similar a un directorio pero en un servidor
un repositorio tiene por default una rama llamada master.
Si quieres hacer cambios primero tienes que sacar un rama de master
modificar esta rama y combinarla con master.
pero entonces como se crea una rama.

como se edita una rama
un commit es un cambio salvado, con cada commit va asociado un
mensaje que da informacion asociada al cambio.Poner mensajes en
los commits es muy importante para tener una historia del proyecto.
cual es la diferencia entre salvar un archivo simplemente y un commit
salvar un archivo solamente guarda la informacion de ese archivo.
Un commit es guardar la informacion de TODO el proyecto.Son como
imagenes instantaneas del proyecto
un head hace referencia al commit mas reciente de una branch
una tag es similar a una head osea , hace referencia a una parte
de la historia del proyecto

basicamente que hace el comando checkout
checkout permite que se cambie la rama en la que ahora se esta
trabajando.cuando haces un checkout , el working directory cambia
a donde branch esta apuntando es como moverte entre ramas.

git fetch unicamente es para checar el repositorio original
sin cambiar nada localmente.

pull request
los pull request es proponer tus cambios a la persona encargada
para que decida si fusiona tu rama con la rama master.
cuando haces un pull request se pueden ver las diferencias entre
la rama propuesta y la rama principal
y el paso final es incorporarlas.
Sin embargo, todo esto fue en la pagina de git , ahora como lo puedo
hacer desde mi maquina.
osea ,lo que yo quiero es tener un proyecto que pueda manejar desde
mis dos maquinas a pesar de que la pequena no tiene internet.
del proyecto para
como se unen dos ramas 

los cambios localmente 
las branches son utilizadas para modificar un documento

ademas esto es lo principal pero tambien hay preguntas como
puedo crear una rama de una rama ?
sea M la rama master y B una rama cualquiera entonces si combine las
ramas dandome la rama M' . puedo regresarme a M ?

puedes descargar una copia de un repositorio git en tu maquina con
git clone
si yo tengo un directorio que quiero guardar en git como lo convierto
en un repositorio 
dentro del directorio de trabajo se pone
git init esto crea un repositorio local 
\end{document}