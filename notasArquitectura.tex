\documentclass[12p]{article}
\usepackage{mathtools}
\usepackage{amsmath}
\usepackage{listings}
\usepackage{color}
\newcommand{\rojo}[1]{
  \textcolor{red}{#1}
  }
\begin{document}
\section{programacion en c}
\rojo{cual es la manera adecuada para regresar excepciones}

\subsection{compilacion}
gcc archivo.c -o archivo -lm 

\subsection{implementacion de una cola en c}
para implementar una cola lo primero que tienes que hacer es definirla en un archivo cabecera. Estos archivos son interfaces
que permiten ver lo que el codigo hace pero no como lo hace.
en el archivo cabecera va a ir la declaracion de la estructura y los metodos.
que metodos tiene una cola?
inicializacion,insercion y eliminacion.

definicion de la estructura
la lista va a tener inicio , final y longitud. inicio y final son de tipo nodo , entonces tambien tiene que ser creado el tipo nodo
\begin{lstlisting}
  typedef struct no {
    char * elem ;
    struct no sig;
  }*nodo;
\end{lstlisting}
esta declaracion es similar a 
\begin{lstlisting}
  struct no {
    char* elem;
    struct no sig;
  };
  typedef struct no *nodo;
\end{lstlisting}

pero que hace la declaracion typedef
con la declaracion typedef es posible asignar un nuevo nombre a un tipo.

porque no es .. nodo simplemente en lugar de *nodo
nodo es una estructura y *nodo es la referencia a una estructura osea directamente no puedes guardar en una variable
un nodo sino la referencia a ese nodo.Nota que no estas haciendo un alias para la estructura , estas haciendo un alias para
un apuntador a la estructura.
ejemplo de esto
\begin{lstlisting}
  typedef char *cadena ;
  int main(){
    cadena st1 = "nieves";
    printf("%s\n",st1);
  }
\end{lstlisting}
en realidad se dio un alias no para char sino para los punteros que apuntan a char.

Ahora . declaracion de los metodos.
el primer metodo que se tiene que implementar es la inicializacion
como inicializarias tu. bueno pues para inicializar una lista primero tienes que asignar la longitud a 0 , ... osea tienes apartar memoria
para un nodo y apartar memoria del tamano de un nodo con malloc ...
osea inicio = malloc(sizeof(nodo))
final = malloc(sizeof(nodo)
ademas con estos enunciados malloc(sizeof(nodo)) lo que estarias diciendo es calcula el tamano de nodo , nodo fue definido como un apuntador
a una estructura como un estructura , entonces sizeof te va a regresar el tamano que necesita el apuntador a la estructura
pero no el tamano que necesita la estructura. No deberia funcionar , lo que deberia funcionar es poner en size of el tamano de la estructura
osea sizeof(no) no nodo \rojo{prueba}

c asigna memoria pero no la limpia?
que pasaria si asignas memoria y luego tratas de acceder a ella \rojo{probar}

como funciona la funcion malloc
la funcion malloc aparta un area de memoria del tamano del argumento que se le esta pasando y regresa la direccion de esa
area de memoria 

entonces lo primero que se hace para inicializar una lista es
utilizar malloc para que reserve memoria pasandole como parametro
el nombre de la estructura , no el puntero a la estructura.
luego cada uno de sus campos tiene que ser asignado
el inicio y el son apuntadores a una estructura de tipo nodo
pero no es necesario apartar memoria , la memoria para estos se
va a apartar en el metodo agregar. a la longitud si se le asigna 0

ahora para crear el metodo agregar
este metodo tiene que tomar un puntero a una lista y una cadena o
puntero a caracter y no va a regresar nada
primero , tenemos que estar de acuerdo por donde va a agregar
en este caso va a agregar por la cola .
entonces hay dos casos : cuando tiene solo un elemento la lista
y cuando no tiene elementos
cuando no tiene elementos entonces(en ambos casos se crea el nodo)
el nodo que se creo va a ser el inicio y el final de la lista.cuando
la lista tiene $\leq 1$ entonces , como se va a agregar por el final ,
primero , el siguiente del final va a ser actualizado al nuevo nodo ,
y el final va de la lista va a ser actualizado a el nuevo nodo 

si el argumento de malloc es un puntero entonces cuanta memoria va
a guardar

\end{document}